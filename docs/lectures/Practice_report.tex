% Options for packages loaded elsewhere
\PassOptionsToPackage{unicode}{hyperref}
\PassOptionsToPackage{hyphens}{url}
\PassOptionsToPackage{dvipsnames,svgnames*,x11names*}{xcolor}
%
\documentclass[
]{article}
\usepackage{lmodern}
\usepackage{amssymb,amsmath}
\usepackage{ifxetex,ifluatex}
\ifnum 0\ifxetex 1\fi\ifluatex 1\fi=0 % if pdftex
  \usepackage[T1]{fontenc}
  \usepackage[utf8]{inputenc}
  \usepackage{textcomp} % provide euro and other symbols
\else % if luatex or xetex
  \usepackage{unicode-math}
  \defaultfontfeatures{Scale=MatchLowercase}
  \defaultfontfeatures[\rmfamily]{Ligatures=TeX,Scale=1}
\fi
% Use upquote if available, for straight quotes in verbatim environments
\IfFileExists{upquote.sty}{\usepackage{upquote}}{}
\IfFileExists{microtype.sty}{% use microtype if available
  \usepackage[]{microtype}
  \UseMicrotypeSet[protrusion]{basicmath} % disable protrusion for tt fonts
}{}
\makeatletter
\@ifundefined{KOMAClassName}{% if non-KOMA class
  \IfFileExists{parskip.sty}{%
    \usepackage{parskip}
  }{% else
    \setlength{\parindent}{0pt}
    \setlength{\parskip}{6pt plus 2pt minus 1pt}}
}{% if KOMA class
  \KOMAoptions{parskip=half}}
\makeatother
\usepackage{xcolor}
\IfFileExists{xurl.sty}{\usepackage{xurl}}{} % add URL line breaks if available
\IfFileExists{bookmark.sty}{\usepackage{bookmark}}{\usepackage{hyperref}}
\hypersetup{
  pdftitle={  RMS2  Practice Coursework report (formative, non-assessed)},
  colorlinks=true,
  linkcolor=Maroon,
  filecolor=Maroon,
  citecolor=Blue,
  urlcolor=Blue,
  pdfcreator={LaTeX via pandoc}}
\urlstyle{same} % disable monospaced font for URLs
\usepackage[margin=1in]{geometry}
\usepackage{graphicx,grffile}
\makeatletter
\def\maxwidth{\ifdim\Gin@nat@width>\linewidth\linewidth\else\Gin@nat@width\fi}
\def\maxheight{\ifdim\Gin@nat@height>\textheight\textheight\else\Gin@nat@height\fi}
\makeatother
% Scale images if necessary, so that they will not overflow the page
% margins by default, and it is still possible to overwrite the defaults
% using explicit options in \includegraphics[width, height, ...]{}
\setkeys{Gin}{width=\maxwidth,height=\maxheight,keepaspectratio}
% Set default figure placement to htbp
\makeatletter
\def\fps@figure{htbp}
\makeatother
\setlength{\emergencystretch}{3em} % prevent overfull lines
\providecommand{\tightlist}{%
  \setlength{\itemsep}{0pt}\setlength{\parskip}{0pt}}
\setcounter{secnumdepth}{-\maxdimen} % remove section numbering
\usepackage{fontspec}
\setmainfont{Roboto}
\usepackage{booktabs}
\usepackage{longtable}
\usepackage{array}
\usepackage{multirow}
\usepackage{wrapfig}
\usepackage{float}
\usepackage{colortbl}
\usepackage{pdflscape}
\usepackage{tabu}
\usepackage{threeparttable}
\usepackage{threeparttablex}
\usepackage[normalem]{ulem}
\usepackage{makecell}
\usepackage{xcolor}

\title{\includegraphics[width=0.51in,height=\textheight]{images/dapr1.png}
\protect \linebreak RMS2 \protect \linebreak Practice Coursework report
(formative, non-assessed)}
\usepackage{etoolbox}
\makeatletter
\providecommand{\subtitle}[1]{% add subtitle to \maketitle
  \apptocmd{\@title}{\par {\large #1 \par}}{}{}
}
\makeatother
\subtitle{Department of Psychology, The University of Edinburgh}
\author{}
\date{\vspace{-2.5em}Academic year 2020-2021}

\begin{document}
\maketitle

\hypertarget{key-dates}{%
\section{Key Dates}\label{key-dates}}

\textbf{Coursework set}: 17:00, Friday 23rd October 2020\\
\textbf{Coursework due}: 12noon, Thursday \textbf{12th May} November
2020

\hypertarget{instructions}{%
\section{Instructions}\label{instructions}}

You need to produce a report answering the assignment questions detailed
on the following pages. You are provided with the description of a
research project and an accompanying data set. Your task is to describe
and analyse the data in order to provide answers to the research
questions. Analyses will draw on the methodologies we have discussed in
lectures and labs.

Try to write your report as if you are writing a paper, or your
dissertation - i.e., write an analysis section and a results section.

The analyses section should detail the appropriate analyses you
undertook and how they will provide answers to the research questions.
The results section should present and discuss your findings, utilising
graphics where necessary to illustrate your points. Analyses will draw
on the methodologies we have discussed in lectures and labs.

\textbf{Please note that this is an individual assignment and you are
expected to work on your own with respect to both R code and report.}

\hypertarget{grading}{%
\subsection{Grading}\label{grading}}

This report will not be graded. Formative feedback will be provided.

\hypertarget{queries-concerning-the-task}{%
\subsection{Queries concerning the
task}\label{queries-concerning-the-task}}

This document contains a basic overview of the task and of how to submit
it. If you have any questions concerning the coursework report, we ask
that you post them on the designated section of the on-line discussion
board on Learn. If you have a question, it is likely your classmates may
have the same question. Before posting a question, please check the
on-line board in case it has already been answered.

\pagebreak

\hypertarget{helpful-hints-for-writing-reports}{%
\paragraph{Helpful hints for writing
reports:}\label{helpful-hints-for-writing-reports}}

\begin{itemize}
\tightlist
\item
  Important things that your report should clearly describe:

  \begin{itemize}
  \tightlist
  \item
    your decisions in cleaning the data
  \item
    your statistical approach to answering each question (in detail -
    for instance, explain model structures)
  \item
    your results
  \item
    your interpretation of these results and how this answers the
    question.
  \end{itemize}
\item
  A reader of your report should be able to more or less replicate your
  analyses \textbf{without} referring to your R code.
\end{itemize}

\hypertarget{report-formatting}{%
\subsection{Report Formatting}\label{report-formatting}}

\begin{itemize}
\tightlist
\item
  Figures and tables should be numbered and captioned, and referred to
  in the text; important statistical outcomes should be summarised in
  the text.
\item
  Reporting should follow APA 7th Edition guidelines for the
  presentation of tables, figures, and statistical results. 
\item
  Your report should be a maximum of 6 sides of single-spaced A4
  (including tables and figures), in a standard font, size 12, with
  normal 1 inch margins.
\end{itemize}

\hypertarget{documents-to-submit}{%
\subsection{Documents to submit}\label{documents-to-submit}}

You are required to submit two files, a complete report and an
associated code file. There are two formats you can use.

\hypertarget{option-1-rmarkdown}{%
\subsubsection{Option 1: Rmarkdown}\label{option-1-rmarkdown}}

If you choose to write your report using Rmarkdown, please submit the
.rmd file and a compiled HTML.

\hypertarget{option-2-word-or-equivalent-and-r-script}{%
\subsubsection{Option 2: Word (or equivalent) and R
script}\label{option-2-word-or-equivalent-and-r-script}}

You may also submit a word file containing your report, and an
associated R script containing all the code required to reproduce the
analyses included in your report.

\hypertarget{submission-instructions-read-carefully}{%
\subsection{Submission instructions: Read
carefully!}\label{submission-instructions-read-carefully}}

Please submit both files on-line via the Turnitin link on the LEARN page
for RMS2. The submission link will be within the Assessments tab and
will become available after you click on the ``Own work declaration''
link.

\textbf{Please include your exam number in your filename. For instance,
\emph{B123405.Rmd}}

Prior to submitting, check the following:

\begin{itemize}
\tightlist
\item
  Does it compile? (i.e., can you \texttt{knit} your Rmarkdown document
  into either .html or .pdf?)\\
\item
  Does the code run line-by-line without throwing any errors?
\end{itemize}

\pagebreak

\hypertarget{coursework-task}{%
\section{COURSEWORK TASK}\label{coursework-task}}

A research team are interested improving driver safety. They are
interested in how risky driving behaviour is influenced by
characteristics of individuals and their interactions with environmental
stimuli and substance use. One of the studies in the research project
involves the relation between alcohol and marijuana use and risky
behaviour.

In this study, participants are randomly assigned to one of four
conditions (control, low alcohol, high alcohol and marijuana; n=20). All
participants were male, had been driving for between 3-5 years, drove on
average for 10 hours per week (self-reported) and had never been
convicted by the police for any driving offenses. Participants were
recruited evenly across two age groups (young 18-25) and old (50-60),
with 10 young and 10 old participants in each of the experimental
groups.

Participants were asked to complete a 20 minute driving simulation in
which various dangers and situations arise in which participants must
make decisions on how to act quickly. The decisions participants make
are scored based on the degree of risky behaviour they represent. The
research team have also measured a host of other variables, some
demographic, some as focal covariates.

The research team would like to answer the following questions:

\begin{enumerate}
\def\labelenumi{\arabic{enumi}.}
\tightlist
\item
  Are there differences in risky driving behaviour across experimental
  conditions?
\item
  Do the differences (identified in 1) remain the same after controlling
  for weight and impulsivity?
\item
  Does the effect of impulsivity on risking driving behaviour change as
  a function of age?
\end{enumerate}

Devise an analysis strategy based on the questions above and the data
provided, complete the analysis, and write up the strategy, results and
a short discussion following the guidelines in the next section.

Using the data described in Table \ref{tab:tab1} and available (in
\textbf{.csv} format) at \href{url}{LINK}, conduct, interpret and write
up a set of analyses that answer the researcher's research question.

\begin{table}[H]

\caption{\label{tab:unnamed-chunk-1}\label{tab:tab1}EXAMPLE - Data Dictionary}
\centering
\begin{tabular}[t]{l>{\raggedright\arraybackslash}p{15cm}>{}p{15cm}}
\toprule
variable & description\\
\midrule
\rowcolor{gray!6}  ID & Unique participant identifier\\
Child\_gend & Gender with 0=female, 1=male\\
\rowcolor{gray!6}  Dep1 & ‘Anhedonia’ measured on a 5-point Likert scale\\
Dep2 & ‘Lack of hope’ measured on a 5-point Likert scale\\
\rowcolor{gray!6}  Dep3 & ‘Lack of energy’ measured on a 5-point Likert scale\\
\addlinespace
Dep4 & ‘Sad without reason’ measured on a 5-point Likert scale\\
\rowcolor{gray!6}  Dep5 & ‘Tired all the time’ measured on a 5-point Likert scale\\
Dep6 & ‘Feeling worthless’ measured on a 5-point Likert scale\\
\bottomrule
\end{tabular}
\end{table}

\end{document}
